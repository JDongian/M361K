\documentclass{article}
\usepackage{amssymb}
\usepackage{amsmath}
\usepackage{centernot}

\begin{document}

\title{M361K\: Homework 2}
\author{Joshua Dong}
\date{\today}
\maketitle

\subsection{2.3.1}
Let $S_1 = \{x \in \mathbb{R}: x \ge 0\}$. Show $S_1$ has lower bounds, but
no upper bounds. Show that $\inf{S_1} = 0$.
\\Assume by that an upper bound, $v \in \mathbb{R}$, exists.
\\$v \ge 0$ (since $\exists x \in S_1$ where $x \ge 0$, and an upper bound
must be greater than any given element of the set).
\\$v+1 \ge 1 \ge 0$.
\\$v+1 \ge 0 \rightarrow v+1 \in S_1$. Therefore $v$ is not an upper bound,
contradiction.
\\$\therefore \nexists v \in \mathbb{R}$ where $v$ is a lower bound of
$S_3$. 
\\
\\0 is a lower bound of $S_1$, by definition of $S_1$.
\\Let $t \in \mathbb{R}$ where $t > 0$.
\\$t > 0$ and $2 > 0.
\\\therefore \frac{t}{2} > 0.
\\t > \frac{t}{2} > 0.
\\\therefore$ t is not a lower bound.
\\Thus $\inf{S_1} = 0$.

\subsection{2.3.3}
Let $S_3 = \{\frac{1}{n}: n \in \mathbb{N}\}$. Show that $\sup{S_3} = 1$ and
$\inf{S_3} \ge 0$ (Archemedian property 2.4).
\\First we show that $\frac{1}{n} > \frac{1}{n+1}:
\\\frac{1}{n} - \frac{1}{n+1} = \frac{(n+1)-n}{n(n+1)} = \frac{1}{n(n+1)}.
\\1, n, (n+1) > 0.
\\\therefore \frac{1}{n(n+1)} > 0.
\\\therefore \frac{1}{n} > \frac{1}{n+1}$.
\\This means that the smallest $n \in \mathbb{N}$ will produce the greatest
$\frac{1}{n}$.
\\$\therefore \sup{S_3} = \frac{1}{1} = 1$.
\newpage
\noindent Now we show that the infinimum is 0 because $\frac{1}{n}$
can become arbitrarily close to 0.
\\$1,n > 0 \rightarrow \frac{1}{n} > 0.
\\\therefore 0$ is a lower bound for $S_3$.
\\$\therefore \exists w = \inf{S}, w \ge 0.
\\\forall \varepsilon > 0 \;\;\; \frac{1}{\varepsilon} \in \mathbb{R}
\rightarrow \exists n \in \mathbb{N}$ such that $\frac{1}{\varepsilon} < n
\rightarrow \frac{1}{n} < \varepsilon
\\0 \le w \le \frac{1}{n} < \varepsilon
\\\forall \varepsilon > 0 \;\;\; 0 \le w < \varepsilon
\\\therefore w = 0$.
\\$\therefore \inf{S_3} = 0$.

\subsection{2.3.4}
Let $S_4 = \{1-\frac{(-1)^n}{n}: n \in \mathbb{N}\}$. Find $\inf{S_4}$ and
$\sup{S_4}$.
\\Let $S_4 = 1+S_5 = 1+\{-\frac{(-1)^n}{n}: n \in \mathbb{N}\}.
\\-\frac{(-1)^n}{n} \le |-\frac{(-1)^n}{n}| \forall n \in \mathbb{N}.
\\\therefore \sup{S_5} \le \sup{\{|-\frac{(-1)^n}{n}|: n \in \mathbb{N}\}}.
\\\therefore 1$ is an upper bound on $S_5.
\\1$ is an element of $S_5$ (when $n=1).
\\\therefore 1 = \sup{S_5}.
\\\therefore 2 = \sup{S_4}.
\\
\\$By symmetry, $\inf{S_5} \ge
\inf{\{-|-\frac{(-1)^n}{n}|: n \in \mathbb{N}\}}.
\\$Observe that this set is increasing. This means that the first value that
is in both sets will be the infinimum of $S_5.
\\-1 \not\in \inf{\{-\frac{(-1)^n}{n}: n \in \mathbb{N}\}}.
\\-\frac{1}{2} \in \inf{\{-\frac{(-1)^n}{n}: n \in \mathbb{N}\}}.
\\\therefore -\frac{1}{2} = \inf{S_5}.
\\\therefore \frac{1}{2} = \inf{S_4}$.

\subsection{2.3.6}
Let $S$ be a nonempty subset of $\mathbb{R}$ that is bounded below.
\\Prove that $\inf{S} = -\sup{\{-s: s \in S\}}$.
\\
\\By definition of infinimum, $\forall s \in S$, $\inf{S} \le s$ and
$\nexists$ a lower bound $m$ such that $m > \inf{S}$.
\\$\therefore -\inf{S} > -s \;\;\; \forall {-s} \in S$.
\\This implies $-\inf{S}$ is an upper bound for $\{-s: s \in S\}$.
\\
\\Since $\nexists$ a lower bound $m$ such that $m > \inf{S}$,
\\$m \le \inf{S} \;\;\; \forall m$ where m is a lower bound.
\\Using similar logic to the previous argument,
\\$m$ is a lower bound $\rightarrow -m > -s \;\;\; \forall {-s} \in S
\rightarrow -m$ is an upper bound.
\\$\therefore -m \ge -\inf{S} \;\;\; \forall -m$ where $-m$ is an upper
bound.
\\
\\$\therefore -\inf{S} = \sup{\{-s: s \in S\}}$, by definition of supremum.
\\$\therefore \inf{S} = -\sup{\{-s: s \in S\}}$.

\subsection{2.3.7}
If a set $S \subseteq \mathbb{R}$ contains one of its upper bounds, show that
this upper bound is the supremum of $S$.
\\We need only prove that any upper bound of $S$ must be greater than or
equal to its higest element:
\\Let $v$ be an upper bound of $S$. We want to show that the greatest
element, $u$, is less than or equal to $v$.
\\Assume the contrary, that is, $u > v$. Since $u$ is an element of $S$, $v$
is not an upper bound, contradiction.
\\Therefore if a set $S \subseteq \mathbb{R}$ contains one of its upper
bounds, this upper bound is the supremum of $S$.

\subsection{2.3.8}
Let $S \subseteq \mathbb{R}$ be nonempty. Show that $u \in \mathbb{R}$ is
an upper bound of $S$ iff $(t \in \mathbb{R}$ and $t > u)
\rightarrow t \not\in S$.
\\$(t \in \mathbb{R}$ and $t > u) \rightarrow t \not\in S
\\t \not\in S \;\;\; \forall t \in \mathbb{R}$ where $t > u.
\\\neg(\exists t \in \mathbb{R}$ where $t > u$ and $t \in S).
\\t \le u \;\;\; \forall t \in \mathbb{R}$ where $t \in S.
\\t \le u \;\;\; \forall t \in S$.
\\This is the definition of an upper bound.

\end{document}
