\documentclass{article}
\usepackage{amssymb}
\usepackage{amsmath}
\usepackage{centernot}

\begin{document}

\title{M361K\: Homework 2}
\author{Joshua Dong}
\date{\today}
\maketitle

\subsection{2.1.1}
$a, b \in \mathbb{R}$. Prove:
\\a) $a+b = 0 \rightarrow b = -a
\\a+b = 0
\\a+b+(-a) = 0+(-a)
\\b+a+(-a) = -a $ by A1 $
\\b+(a-a) = -a $ by A2 $
\\b = -a
\\\therefore a+b = 0 \rightarrow b = -a$.
\\b) $-(-a) = a
\\-(-a) = -1 \cdot (-1 \cdot a) =
\\$ by M2 $ (-1 \cdot -1) \cdot a =
\\1 \cdot a = a
\\\therefore -(-a) = a$.
\\c) $(-1)a = -a
\\(-1)a = -(1a) = -a $ by M2 and M3, respectively $
\\\therefore (-1)a = -a$.
\\d) $(-1)(-1) = 1
\\(-1)(-1) = -(-1) = 1 $ by b) $
\\\therefore (-1)(-1) = 1$.

\subsection{2.1.2}
$a, b \in \mathbb{R}$. Prove:
\\a) $-(a+b) = (-a) + (-b)
\\-(a+b) = (-a) + (-b) $ by the distributive law of multiplication of reals $
\\\therefore -a+-b = (-a) + (-b)$.
\\b) $(-a) \cdot (-b) = a \cdot b
(-a) \cdot (-b) = (-1 \cdot a) \cdot (-1 \cdot b) =
\\$ by M1 and M2 $ (-1 \cdot -1) \cdot (a \cdot b) = 
\\a \cdot b
\\\therefore (-a) \cdot (-b) = a \cdot b$.
\\c) $\frac{1}{-a} = -(\frac{1}{a})
\\\frac{1}{-a} =
\\\frac{1}{-1 \cdot a} = 
\\$ by M1 $ -1 \cdot \frac{1}{a} = 
\\-(\frac{1}{a}).
\\\therefore \frac{1}{-a} = -(\frac{1}{a})$.
\\d) $b \neq 0 \rightarrow -(\frac{a}{b}) = \frac{-a}{b}
\\$ Since $b^{-1}$ exists, $
\\-(\frac{a}{b}) = -1 \cdot \frac{a}{b} =
\\$ by M1 $ \frac{-1 \cdot a}{b} =
\\\frac{a}{b}
\\\therefore b \neq 0 \rightarrow -(\frac{a}{b}) = \frac{-a}{b}$.

\subsection{2.1.4}
$a \in \mathbb{R}$ and $a \cdot a = a$. Prove $a \in \{0, 1\}$.
\\Let $n \in \mathbb{R}$ where $a = n+1$.
\\$a \cdot a = a
\\(1+n) \cdot (1+n) = (1+n)
\\n^2 + 2n +1 = n + 1
\\n^2 + n = 0
\\n(n+1) = 0$.
\\The above is only true if $n \in \{-1, 0\}
\\\therefore \; a \in \{-1+1, 0+1\} = \{0, 1\} $

\subsection{2.1.5}
Show $a \neq 0$ and $b \neq 0 \rightarrow \frac{1}{ab} =
(\frac{1}{a})(\frac{1}{b})$.
\\$a \neq 0$ and $b \neq 0
\\\therefore \exists a^{-1}, \exists b^{-1}.
\\$ Let $n \in \mathbb{R}$ where $\frac{1}{ab} = n.
\\$ Note that $n \cdot ab = \frac{1}{ab} \cdot ab = (ab)^{-1} \cdot ab = 1.
\\\frac{1}{ab} = n
\\ab \cdot \frac{1}{ab} = ab \cdot n
\\a^{-1}b^{-1}ab \cdot \frac{1}{ab} = a^{-1}b^{-1}ab \cdot n =
a^{-1}b^{-1} \cdot 1.
\\\therefore \frac{1}{ab} = a^{-1}b^{-1}
\\\therefore a \neq 0$ and $b \neq 0 \rightarrow \frac{1}{ab} =
(\frac{1}{a})(\frac{1}{b})$.

\subsection{2.1.6}
Show $\nexists s \in \mathbb{R}$ such that $s^2 = 6$.
\\Suppose, on the contrary, that $p$ and $q$ are integers such that
$(\frac{p}{q})^2 = 6$. We may assume that $p$ and $q$ are positive and
$p \perp q$. Since $p^2 = 6q^2$, we see that $2 \mid p^2$ and $3 \mid p^2$,
thus $2 \mid p$ and $3 \mid p$ and $6 \mid p$. $p \perp q \rightarrow
6 \nmid q$.
\\Since $6 \mid p$, then $p = 6m$ for some $m \in \mathbb{N}$, and hence
$36m^2 = 6q^2$, so that $6m^2 = q^2. \therefore 6 \mid q^2$, and it follows
that $2 \mid q^2$ and $3 \mid q^2$, thus $2 \mid q$ and $3 \mid q$ and
$6 \mid q$. Since $(\frac{p}{q})^2 = 6 \rightarrow 6 \mid q$ and $6 \mid p$,
but $q \perp p$, the hypothesis must be false.

\subsection{2.2.1}
$a,b \in \mathbb{R}. \;\; b \neq 0.$ Show:
\\a) $|a| = \sqrt{(a^2)}$
\\$|a|^2 = a^2$ by 2.2.2.b.
\\$\sqrt{|a|^2} = \sqrt{a^2}$.
\\$|a| = \sqrt{a^2}$, since $|a| > 0.
\\\therefore |a| = \sqrt{(a^2)}$.
\\
\\b) $|\frac{a}{b}| = \frac{|a|}{|b|}$
\\5 cases:
\\1) $a < 0, b < 0$
\\2) $a < 0, b > 0$
\\3) $a > 0, b > 0$
\\4) $a > 0, b < 0$
\\5) $a = 0, b \neq 0$
\\
\\1) $a < 0, b < 0
\\|a| = -a, |b| = -b.
\\\frac{|a|}{|b|} = \frac{-a}{-b} = \frac{a}{b}.
\\b < 0. \;\; 1 > 0. \;\; \therefore b^{-1} < 0.
\\\frac{a}{b} = ab^{-1} > 0$, since the product of two negatives is positive.
\\$\therefore |\frac{a}{b}| = \frac{a}{b}.
\\\frac{|a|}{|b|} = \frac{a}{b} = |\frac{a}{b}|.
\\\therefore a < 0, b < 0 \rightarrow |\frac{a}{b}| = \frac{|a|}{|b|}$.
\\
\\2) $a < 0, b > 0
\\|a| = -a, |b| = b.
\\\frac{|a|}{|b|} = \frac{-a}{b} = -\frac{a}{b}.
\\b > 0. \;\; 1 > 0. \;\; \therefore b^{-1} > 0.
\\\frac{a}{b} = ab^{-1} < 0$, since the product of negative and positive is
negative.
\\$\therefore |\frac{a}{b}| = -\frac{a}{b}.
\\\frac{|a|}{|b|} = -\frac{a}{b} = |\frac{a}{b}|.
\\\therefore a < 0, b > 0 \rightarrow |\frac{a}{b}| = \frac{|a|}{|b|}$.
\\
\\3) $a > 0, b > 0
\\|a| = a, |b| = b.
\\\frac{|a|}{|b|} = \frac{a}{b} = \frac{a}{b}.
\\b > 0. \;\; 1 > 0. \;\; \therefore b^{-1} > 0.
\\\frac{a}{b} = ab^{-1} > 0$, since the product of two positives is positive.
\\$\therefore |\frac{a}{b}| = \frac{a}{b}.
\\\frac{|a|}{|b|} = \frac{a}{b} = |\frac{a}{b}|.
\\\therefore a > 0, b > 0 \rightarrow |\frac{a}{b}| = \frac{|a|}{|b|}$.
\\
\\4) $a > 0, b < 0
\\|a| = a, |b| = -b.
\\\frac{|a|}{|b|} = \frac{-a}{b} = -\frac{a}{b}.
\\b < 0. \;\; 1 > 0. \;\; \therefore b^{-1} < 0.
\\\frac{a}{b} = ab^{-1} < 0$, since the product of negative and positive is
negative.
\\$\therefore |\frac{a}{b}| = -\frac{a}{b}.
\\\frac{|a|}{|b|} = -\frac{a}{b} = |\frac{a}{b}|.
\\\therefore a > 0, b < 0 \rightarrow |\frac{a}{b}| = \frac{|a|}{|b|}$.
\\
\\5) $a = 0, b \neq 0.
\\\frac{|a|}{|b|} = 0 = |0| = |\frac{a}{b}|.
\\\therefore a = 0, b \neq 0 \rightarrow |\frac{a}{b}| = \frac{|a|}{|b|}$.
\\
\\$\therefore |\frac{a}{b}| = \frac{|a|}{|b|} \forall a,b \in \mathbb{R},
b \neq 0$.

\subsection{2.2.5}
If $a < x < b$ and $a < y < b$, show that $|x - y| < b - a$. Interpret this
geometrically.
\\$a < x < b, a < y < b.
\\-a > -y  > -b.
\\a - a > a - y > a - b.
\\a - b < a - y.
\\
\\-a > -y  > -b.
\\b - a > b - y > b - b.
\\b - y < b - a.
\\
\\a < x < b
\\a - y < x - y < b - y
\\a - b < a - y < x - y < b - y < b - a.
\\a - b < x - y < b - a.
\\
\\x - y > 0 \rightarrow |x - y| = x - y \rightarrow |x - y| < b - a.
\\x - y < 0 \rightarrow |x - y| = -(x - y).
\\-(x - y) < -(a - b) = b - a.
\\\therefore x - y < 0 \rightarrow |x - y| < b - a.
\\\therefore |x - y| < b - a$.
\\Geometrically, this means that any two points selected within a range will
not have a distance greater than the span of the range.

\newpage

\subsection{2.2.16}
Let $\varepsilon > 0$ and $\delta > 0$, and $a, b \in \mathbb{R}$. Show that
$V_\varepsilon(a) \cap V_\delta(a)$ and $ V_\varepsilon(a) \cup V_\delta(a)$
are $\gamma$-neighborhoods of $a$ for appropriate values of $\gamma$.
\\$V_\varepsilon(a) \cap V_\delta(a) =
\{x \in \mathbb{R} \mid |x-a| < \varepsilon\} \cap
\{x \in \mathbb{R} \mid |x-a| < \delta\} =
\\\{x \in \mathbb{R} \mid |x-a| < min(\varepsilon, \delta)\} = 
V_{min(\varepsilon, \delta)}(a).
\\min(\varepsilon, \delta) \in \{\varepsilon, \delta\} \subseteq \mathbb{R}.
\\\therefore \exists \gamma \in \mathbb{R}$ where $V_\gamma(a) \subseteq
V_\varepsilon(a) \cap V_\delta(a)$.
\\
\\$V_\varepsilon(a) \cup V_\delta(a) =
\{x \in \mathbb{R} \mid |x-a| < \varepsilon\} \cup
\{x \in \mathbb{R} \mid |x-a| < \delta\} =
\\\{x \in \mathbb{R} \mid |x-a| < max(\varepsilon, \delta)\} = 
V_{max(\varepsilon, \delta)}(a).
\\max(\varepsilon, \delta) \in \{\varepsilon, \delta\} \subseteq \mathbb{R}.
\\\therefore \exists \gamma \in \mathbb{R}$ where $V_\gamma(a) \subseteq
V_\varepsilon(a) \cup V_\delta(a).
\\
\\\therefore V_\varepsilon(a) \cap V_\delta(a)$ and $ V_\varepsilon(a) \cup
V_\delta(a)$ are $\gamma$-neighborhoods of $a$ for appropriate values of
$\gamma$.


\subsection{2.2.17}
Show that $a, b \in \mathbb{R}, a \neq b \rightarrow \exists \;
\varepsilon$-neighborhoods $U$ of $a$ and $V$ of $b$ such that $U \cap
V = \emptyset$.
\\Let $\alpha \in \mathbb{R}$ where $V_\alpha(a) = U$ and let $\beta \in
\mathbb{R}$ where $V_\beta(b) = V$.
\\Without loss of generality, let $a < b$.
\\If we can show that $max(V_\alpha(a)) < min(V_\beta(b))$, then $U$ and $V$
are disjoint, as any intersection between them would contain the points
$max(V_\alpha(a))$ and $min(V_\beta(b))$.
\\Suppose $\alpha = \beta = \frac{b-a}{3}$.
\\Then $\max(V_\alpha(a)) = a + \alpha = a + \frac{b-a}{3}$ and
$\min(V_\beta(b)) = b - \beta = b - \frac{b-a}{3}.
\\
\\max(V_\alpha(a)) - min(V_\beta(b)) =
a + \frac{b-a}{3} - (b - \frac{b-a}{3}) = 
a - b + \frac{2(b-a)}{3} = 
(a-b)(1 - \frac{2}{3}) =
(a-b)(\frac{1}{3}).
\\(a-b)(\frac{1}{3}) < 0$ since the product of positive and negative is
negative.
\\$\therefore max(V_\alpha(a)) < min(V_\beta(b))
\\\therefore U$ and $V$ are disjoint.
\\$\therefore U \cap V = \emptyset$ when $\alpha = \beta = \frac{b-a}{3}.
\\\therefore a, b \in \mathbb{R}, a \neq b \rightarrow \exists \;
\varepsilon$-neighborhoods $U$ of $a$ and $V$ of $b$ such that $U \cap
V = \emptyset$.

\end{document}
